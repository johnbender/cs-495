\documentclass[pdftex,12pt,a4paper]{report}
\usepackage{hyperref}
\usepackage{bookmark}
\makeatletter
\renewcommand\@seccntformat[1]{}
\makeatother

\begin{document}
\section{Riding the Bus in LA Self-Critique}
\subsubsection{John Bender, October 31, 2013}

On Tuesday, October 31st I led a brief discussion about ridding the Bus in Los Angeles. My goal was to provide useful information for my fellow graduate students about the local bus systems as most aren't from the area.

I started the presentation by getting a feel for the attendees knowledge of the bus system. I asked how many owned cars, how many had ridden the bus, and how many were planning to ride the bus. Nearly everyone had ridden the bus or planned to so it was clear that the information would be useful.

Next, I asked the class to list the most important bus systems that service UCLA, where their lines run, and how the transfers between the various systems work (or don't work as is often the case). Toward the end I was asked to explain the purpose of transfers which was unexpected.

Finally I asked the attendees to brainstorm on how a bus rider might deal with an attempted theft of their bike when riding with it on the bus. This problem is entirely accessible to anyone who can visualize the layout of a bus so there was a diversity of useful answers

After the presentation I received a lot of very positive feedback. Generally everyone enjoyed the presentation and my enthusiasm kept it interesting for the duration. My attempts at levity were well received which was encouraging given the diversity of the crowd and the feedback reflected this positive reaction. Additionally the attendees appreciated the handout which contained useful additional information about where to catch buses on campus, and how to get discounts when riding the bus.

There were also some insightful critiques of my presentation. Most importantly, the talk felt disjoint because of the problem solving activity at the end. Additionally my hand writing on the chalk board wasn't particularly clear because I was writing too quickly which was a distraction especially for the international attendees. In the same vein my chalkboard layout turned out poorly due to the unplanned-for addition of a written definition for transfers. Finally it was suggested that the transfer information isn't as useful as some of the items that were included on the handout and that it could have been swapped or simply left there as supplemental information.

Personally I think being clear about my intentions and the purpose of the presentation was the biggest mistake. I simply didn't account for the audiences bias toward problem solving presentations given what had come before. I think that left many of the attendees out of the presentation due to confusion which was unfortunate.
\end{document}
